\section{Manufacture Process}
This section will describe the full pcarefully rocess of developing the step up converter PCB\@. That is simulation, prototype making and printing.\
\subsection{Design and simulation}
Firstly, simulations were made to experimentally come up with a theoretical solution to the problem. In this stage, total freedom was at hand which made the ability of finding errors and differentiate good solutions from bad ones simpler. The simulations were made in Multisim using the build in function generator and multimeter to measure the outcome. When the desired function had been simulated and verifired, the netlist, which is a description of the connectivity of the circuit, was carefully verified to maintain the desired function going into Ultiboard.
\subsection{PCB layout}
The Multisim file was transfered to Ultiboard where the last design choices were to be made. In Ultiboard, two, for the process of printing crucial file were made. The first one \textsc{Copper Top}, which contains the information of where the printer shall print the silver traces, connecting all the components. Thanks to the netlist from Multisim, Ultiboard automatically connects all components making the process significantly easier. The components were placed as compact as possible, with simplicity in mind. The next file \textsc{Solder Mask Top}, contains information about where the printer shall print the soldering masks that the pfysical components later can be placed and soldered on to.
