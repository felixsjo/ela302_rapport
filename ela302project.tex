\documentclass[conference]{IEEEtran}
\usepackage[utf8]{inputenc}

% correct bad hyphenation here
\hyphenation{op-tical net-works semi-conduc-tor}

\begin{document}

\title{En resa mot Svartåns djupa mörker}

% author names and affiliations
% use a multiple column layout for up to three different
% rdalebsaffiliations
\author{\IEEEauthorblockN{Felix Sjöqvist}
    \IEEEauthorblockA{Mälardalen University\\
        Västerås, Sweden\\
        Email: fst17001@student.mdh.se}
\and
    \IEEEauthorblockN{Olle Olofsson}
    \IEEEauthorblockA{Mälardalen University\\
        Västerås, Sweden\\
        Email: oon17003@student.mdh.se}
}

% make the title area
\maketitle

% As a general rule, do not put math, special symbols or citations
% in the abstract

\begin{abstract}
The abstract goes here.
\end{abstract}

\section{Introduction}
% no \IEEEPARstart
\textit{Explain the subject. What was i studying? Why was this topic important to investigate? What did we know about the topic before I did this study? How will this study dvance new knowledge or new ways of understanding?}
\section{State of the Art}
\textit{The latest and most sophisticated or advanced stage of a technology or science. State of the art if the foundation for determining the methid and methodlogy.}
\section{Hypothesis}
\textit{In scienc, a hypothesis is an idea or explanation that you then tesr through study and experimentation. Outside science, a theory or quess can also be called a hypothesis}
\section{Problem formulation}
\textit{The problm formulation is defined upon hypothesis to define the problem or problems for the thesis}
\section{Research questions}
\textit{A research question guides and centers your research. It should be clear and focusd, as well as synthesize multiple sources to present your unque argument. RQ should be furmulat}
%Three & Four\\
\section{Hardware}

\section{Software}

\section{Method}
\textit{How will you test the hypothesis? What methods will be used from the knowledge learned in state of the art?}
\\The PCB was tested using the National Instruments \textit{myDAQ}, by imposing a square wave with following characteristics:
\begin{itemize}
        \item Constant 5V amplitude.
        \item Constant 2.5V positive offset.
        \item Variable frequency $100Hz-10kHz$
        \item Variable duty-cycle $10\% - 90\%$
\end{itemize}
and then measuring the output
\section{Results}
\textit{What are the results your method have given?}

\section{Conclusion}
\textit{Have you provn or disproven the hypothesis? If not, why?}
\section{Discussion}

\section{Future work}
\textit{What is the best way to continue the work?}
% conference papers do not normally have an appendix


% use section* for acknowledgment
\section*{Acknowledgment}


%The authors would like to thank...
\begin{thebibliography}{1}

\bibitem{IEEEhowto:kopka}
H.~Kopka and P.~W. Daly, \emph{A Guide to \LaTeX}, 3rd~ed.\hskip 1em plus
  0.5em minus 0.4em\relax Harlow, England: Addison-Wesley, 1999.

\end{thebibliography}




% that's all folks
\end{document}


