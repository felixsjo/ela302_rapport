\section{Hardware}
A note on replicability: Although the hardware-related equipment used in this paper could be considered to be of a level of sophistication unreachable or undesirable, a substantially more rudementary setup would suffice. For testing purposes, a simple PWM-signal generating device such as a function generator, or even simpler, an Arduino UNO \cite{abhi} could be used. Since the circuit is designed to run on 5V DC, even an ad hoc power supply comprised of a cut USB-cord plugged into a wall adpater for phone chargers, or an unused USB-port, could make due. As for measuring, any old multimeter would do.
\begin{itemize}
    \item The NI MyDAQ is a simplistic and small all-in-one solution providing all the necessary power and measuring capabilities used in this paper. It provides a steady 5V output for power, a flexible, software-controlled signal generator and a multimeter capable of measuring 60V\cite{mydaq}. Far more than our predicted output, and far beyond what our capacitor would be physically able to produce.
    \item The Voltera Printer was used to realize the circuit for testing purposes\cite{voltera}. It is a desktop-sized device designed for rapid PCB-prototyping, allowing for a short idea-to-test pipeline, similar to what FFF 3D-printing technology has come to provide for mechanical applications design.
\end{itemize}
